\documentclass[submit,techreq,noauthor]{eco}	% semi style
\usepackage[dvips]{graphicx}
\usepackage{listings, jlisting} 		% for source code
\usepackage{url}
\usepackage{setspace}
\usepackage{here}
%\setstretch{1.5} % 行間を広くします(資料チェックしてもらうときはコメントを外す)
\lstset{
  basicstyle={\ttfamily},
  identifierstyle={\small},
  commentstyle={\smallitshape},
  keywordstyle={\small\bfseries},
  ndkeywordstyle={\small},
  stringstyle={\small\ttfamily},
  frame={tb},
  breaklines=true,
  columns=[l]{fullflexible},
  numbers=left,
  xrightmargin=0zw,
  xleftmargin=3zw,
  numberstyle={\scriptsize},
  stepnumber=1,
  numbersep=1zw,
  lineskip=-0.5ex
}

\begin{document}

\semino {4/7}					% 年度/回数
\date   {4/11/25/金}				% 平成/月/日/曜日
\title  {Operating System Support for Safe and Efficient Auxiliary Execution}	% タイトル
\author {山下 恭平}				% 氏名

\begin{abstract}
本稿はUSENIX,OSDI'22 に掲載されている論文「Operating System Support
 for Safe and Efficient Auxiliary Execution」\begin{math}^{[1]}\end{math}の内容についてまとめた
ものである.近年のアプリケーションは様々な補助タスクが実行されている.補助タスクとは,アプリ
ケーションが自身のメンテナンス,自己管理を行うタスクのことである.これらのタス
クは、アプリケーションのアドレス空間で実行することで,高い観測性と制御性
を得るが,安全性と性能の問題が発生する.また,補助タスクを別のプロセスで実行す
ると,分離性は高いが,観測性と制御性が劣る.本稿では,この問題を解決するために,
補助タスクに対するサポートとして,orbitと呼ばれるOSの抽象化機能を提案する.

\end{abstract}
\maketitle

\section{はじめに}
運用されているアプリケーションはその実行状況を調査し,最適化し,デバッグし,
制御するために,頻繁にメンテナンスを行う必要がある.かつてはメンテナンスはアプリケ
ーションの管理者が手動で行なっていたが,現在では多くのアプリケーションは,自身
でメンテナンスを行うための補助タスクが行われている.例えばMySQLでは,デッド
ロックを検出するとロールバックを行う機能が存在する.\begin{math}^{[2]}\end{math}
\indent 既存のOSに搭載されているプロセスやスレッドといった抽象化機能は
メインタスクの実行に適した設計がされており,補助タスクの実行には適していない.
そのため,開発者は,分離は強いが観測と制御が非常に限定される(別プロセス)か,
観測と制御は強いが分離はほとんどできない(スレッド)かのどちらかを選ばざる
を得ない.この問題を解決するために,OSの補助タスクに対するサポートを提供する
orbit抽象化を提案する.\\
\indent orbitタスクは,協力な分離を提供する.同時に,状態同期機能により
メインタスクを観測することも可能である.orbitのプロトタイプはLinux kernel 
5.4.91に実装された.orbitの評価を行うためにMySQL,Apacheを含む6つの大規模
アプリケーションから,7つの補助タスクを抽出し,orbitに移行することに成功した.
また,全てのケースでアプリケーションはフォールトから保護されることがわかった.
分離のコストを測定した所.orbitバージョンアプリケーションでは,中央値で3.3
\%のオーバーヘッドが発生した.\\
\indent 本稿では2章でこの研究が行われる背景について述べ,3章でorbitの詳細な
説明を行う,4章でorbitの性能評価を行い,最後の5章では全体のまとめる.

\section{研究の背景}


\section{orbitについて}


\section{orbitの性能評価}


\section{終おわりに}



% 参考文献はここに記述
\begin{thebibliography}{99}
  \bibitem{orbit} Yuzhuo Jing , Peng Huang : Operating System 
  Support for Safe and Efficient Auxiliary Execution , 16th 
  USENIX Symposium on Operating Systems Design and Implementation ,
   pp.633 - 648 (2022)
  \bibitem{MySQL} Oracle. MySQL's deadlock detection.\\
  \url{https://dev.mysql.com/doc/refman/8.0/en/innodb-deadlock-detection.html}
  (11/25/2022)
\end{thebibliography}

\end{document}
